\documentclass[11pt]{article}
\usepackage[italian]{babel}
\usepackage[utf8]{inputenc}	% Para caracteres en español
\usepackage{amsmath,amsthm,amsfonts,amssymb,amscd}
\usepackage{multirow,booktabs}
\usepackage[table]{xcolor}
\usepackage{fullpage}
\usepackage{lastpage}
\usepackage{enumitem}
\usepackage{fancyhdr}
\usepackage{mathrsfs}
\usepackage{wrapfig}
\usepackage{graphicx}
\graphicspath{ {./images/} }
\usepackage{setspace}
\usepackage{calc}
\usepackage{multicol}
\usepackage{cancel}
\usepackage[retainorgcmds]{IEEEtrantools}
\usepackage[margin=3cm]{geometry}
\usepackage{amsmath}
\newlength{\tabcont}
\setlength{\parindent}{0.0in}
\setlength{\parskip}{0.05in}
\usepackage{empheq}
\usepackage{framed}
\usepackage[most]{tcolorbox}
\usepackage{xcolor}
\usepackage{FiraSans}
\colorlet{shadecolor}{orange!15}
\parindent 0in
\parskip 12pt
\geometry{margin=1in, headsep=0.25in}
\theoremstyle{definition}
\newtheorem{defn}{Definizione}
\newtheorem{reg}{Regola}
\newtheorem{oss}{Osservazione}
\newtheorem{exer}{Esecizio}
\newtheorem{dimo}{Dimostrazione}
\newtheorem{note}{Nota}
\newtheorem{thm}{Teorema}[section] % reset theorem numbering for each chapter
\theoremstyle{plain}
\newcommand{\restr}[2]{%
\mathchoice{%
\restriction{#1}{#2}{\displaystyle}%
}{%
\restriction{#1}{#2}{\textstyle}%
}{%
\restriction{#1}{#2}{\scriptstyle}%
}{%
\restriction{#1}{#2}{\scriptscriptstyle}%
}%
}

\newlength{\totbarheight}
\newlength{\bardepth}




\begin{document}

\title{Elaborato Parte 1}
\author{Guglielmo Bartelloni\\Ilaria Rocchi}
\maketitle

\thispagestyle{empty}

\begin{exer}
Per dimostrare la seguente:
\[
	\frac{-f(x+2h)+8f(x+h)-8f(x-h)+f(x-2h)}{12h}=f'(x)+O(h^{4})
.\] 

Eseguiamo lo sviluppo di taylor delle:
\[
	f(x+2h)=f(x)+2hf'(x)+2h^{2}f''(x)+\frac{4}{3}h^{3}f^{3}(x)+\frac{2}{3}h^{4}f^{4}(x)+O(h^{5})
.\] 
\[
	f(x+h)=f(x)+hf'(x)+\frac{1}{2}h^{2}f''(x)+\frac{1}{6}h^{3}f^{3}(x)+\frac{1}{24}h^{4}f^{4}(x)+O(h^{5})
.\] 
\[
	f(x-h)=f(x)-hf'(x)+\frac{1}{2}h^{2}f''(x)-\frac{1}{6}h^{3}f^{3}(x)+\frac{1}{24}h^{4}f^{4}(x)+O(h^{5})
.\] 
\[
	f(x-2h)=f(x)-2hf'(x)+2h^{2}f''(x)-\frac{4}{3}h^{3}f^{3}(x)+\frac{2}{3}h^{4}f^{4}(x)+O(h^{5})
.\] 

Moltiplicando per i coefficienti delle funzioni:
\[
	-f(x+2h)=-f(x)-2hf'(x)-2h^{2}f''(x)-\frac{4}{3}h^{3}f^{3}(x)-\frac{2}{3}h^{4}f^{4}(x)+O(h^{5})
.\] 
\[
	8f(x+h)=8f(x)+8hf'(x)+4h^{2}f''(x)+\frac{4}{3}h^{3}f^{3}(x)+\frac{1}{3}h^{4}f^{4}(x)+O(h^{5})
.\] 
\[
	-8f(x-h)=-8f(x)+8hf'(x)-4h^{2}f''(x)+\frac{4}{3}h^{3}f^{3}(x)-\frac{1}{3}h^{4}f^{4}(x)+O(h^{5})
.\] 

Quindi:
\[
	\frac{-f(x+2h)+8f(x+h)-8f(x-h)+f(x-2h)}{12h}=\frac{12hf'(x)+O(h^{5})}{12h}=f'(x)+O(h^{4})
.\] 

Pertanto l'espressione e' verificata.
\end{exer}

\end{document}



